\documentclass{article}
\usepackage[T2A]{fontenc}
\usepackage[english, russian]{babel}
\title{3DViewer v1.0}
\author{scrimgew, mirrorar, powersbe}

\begin{document}
\maketitle
\section{Описание}
Программа 3DViewer v1.0 выполняет построение каркасной модели по OBJ файлу. 
К построенной модели можно применять аффинные преобразования (поворот, смещение, масштабирование)
К модели можно применять настройки цвета (фона, вершин, линий) типа и размера вершин и линий, настройки сохраняются между запусками программы.
Текущий вид модели можно сохранить в bmp/jpeg файле, а преобразования - в gif.


\section{Инструкция}
\begin{itemize}
\item Для установки программы запустите \textbf{make install}, далее запустите программу скомпилированную в папке /build
\item для создания архива выполните \textbf{make dist}
\end{itemize}
\end{document}